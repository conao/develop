\documentstyle[a4,11pt]{jarticle}
\topmargin=-1.5cm
\textwidth=17cm
\textheight=25cm
\oddsidemargin=-0.5cm
\evensidemargin=-0.5cm

\begin{document}
\begin{center}
  \Large\bf 情報活用演習 課題5
\end{center}

\begin{flushright}
  \begin{tabular}{ll}
    学生番号: & B151235 \\
    氏名:     & 山下 直哉 \\
    提出日:   & 平成27年11月10日 \\
    提出期限: & 平成27年11月17日 \\
  \end{tabular}
\end{flushright}

\begin{center}
  平成27年10月6日 安倍内閣総理大臣記者会見 \\
\end{center}

\section {安倍総理冒頭発言}
新しい「アジア・太平洋の世紀」、いよいよ、その幕開けです。\par
日本とアメリカがリードして、自由、民主主義、基本的人権、法の支配といった価値を共有する国々と共に、このアジア・太平洋に、自由と繁栄の海を築き上げる。TPP協定について、昨日、大筋合意に至りました。\par
かつてない規模の人口8億人、世界経済の4割近くを占める広大な経済圏が生まれます。そして、その中心に日本が参加する。TPPは正に「国家百年の計」であります。\par
TPPは私たちの生活を豊かにしてくれます。\par
それは、貿易に国境がなくなり、世界のバラエティーあふれる商品を安く手に入れることができるというだけではありません。\par
海賊版、偽物の商品を買わされて後悔する。そのようなことは、なくなっていきます。海外に旅行をしたときの電話代も安くなるかもしれません。サイバーの世界を飛び交う皆さんの個人情報もしっかりと守られるようになります。\par
TPPのメリットは、単に関税をなくすだけにとどまりません。安かろう、悪かろうは認めない。サービスから知的財産に至るまで、幅広い分野で品質の高さが正しく評価される、公正なルールを共有し、持続可能な経済圏をつくり上げる野心的な取組であります。\par
TPPは、私たちにチャンスをもたらします。\par
その主役はきらりと光る技を持つ中小・小規模事業者の皆さん。そして個性あふれる、ふるさと名物を持つ地方の皆さんであります。\par
10%近い眼鏡フレームの関税がゼロになる。福井の鯖江ブランドをもっと世界に広げていく絶好の機会であります。\par
日本茶にかかる20%もの関税がゼロになる。静岡や鹿児島が世界有数の茶所と呼ばれる日も近いかもしれません。\par
国によっては30%を超える陶磁器への関税がゼロになる。岐阜の美濃焼や佐賀の有田焼、伊万里焼。日本が誇る伝統の陶磁器は、海外の人たちを魅了するに違いありません。\par
意欲あふれる地方の皆さん、若者の皆さんには、是非TPPという世界の舞台でこのチャンスを最大限いかしてほしいと思います。\par
海外の成長著しいマーケットへと果敢に飛び込む。そうした皆さんには投資を守る新たなルールができます。\par
TPP参加国への投資であれば、その国の政府から技術移転を行ってほしいといった不当な要求が行われることは今後一切なくなります。粘り強く交渉を行った結果、我が国の主張が協定に盛り込まれました。\par
攻めるべきは攻め、守るべきは守る。TPP交渉に臨んで、私は繰り返しこのように述べてまいりました。\par
世界に誇るべき我が国の国民皆保険制度は今後も堅持いたします。食の安全・安心にかかる基準もしっかりと守られます。正当な規制を行うに当たって我が国の主権は全く損なわれることはありません。投資家と国との紛争処理、いわゆるISDSに関して、そのことを確認する規定を盛り込みました。\par
自由民主党がTPP交渉参加に先立って掲げた国民の皆様とのお約束はしっかりと守ることができた。そのことは明確に申し上げたいと思います。\par
中でも、聖域なき関税撤廃は認めることができない。これが交渉参加の大前提であります。\par
特に、米や麦、さとうきび、てんさい、牛肉・豚肉、そして乳製品。日本の農業を長らく支えてきたこれらの重要品目については最後の最後までぎりぎりの交渉を続けました。\par
その結果、これらについて、関税撤廃の例外をしっかりと確保することができました。これらの農産品の輸入が万一急に増えた場合には、緊急的に輸入を制限することができる新しいセーフガード措置を更に設けることも認められました。\par
日本が交渉を積極的にリードすることで、厳しい交渉の中で国益にかなう最善の結果を得ることができた。私はそう考えています。\par
それでも、TPPに入ると農業を続けていけなくなるのではないか。大変な不安を感じておられる方々がたくさんいらっしゃることを私はよく承知をしております。\par
美しい田園風景、伝統あるふるさと、助け合いの農村文化、日本が誇るこうした国柄をこれからもしっかりと守っていく。その決意は今後も全く揺らぐことはありません。\par
私が先頭に立って取り組んでまいります。全ての大臣をメンバーとする「TPP総合対策本部」を設置します。政府全体で責任を持って、できる限りの総合的な対策を実施してまいります。甘利大臣が帰国し、報告を受けた後、具体的な指示を出すこととしています。\par
新たに輸入枠を設定することとなる米についても、必要な措置を講じることで、市場に流通する米の総量は増やさないようにするなど、農家の皆さんの不安な気持ちに寄り添いながら、生産者が安心して再生産に取り組むことができるように、万全の対策を実施していく考えであります。\par
農業こそ国の基であります。しかし、戦後、1,600万人を超えていた農業人口は現在200万人。この70年で8分の1まで減り、平均年齢は66歳を超えました。\par
TPPをピンチではなく、むしろチャンスにしていかなければならない。若者が自らの情熱で新たな地平線を切り開いていくことができる農業へと変えていく起爆剤としなければなりません。\par
TPPでは、多くの国で、農作物にかけられていた関税がなくなります。\par
北海道のメロン、大分の梨、日本にはほかにはないような甘くてジューシーな果物がたくさんあります。新潟にはコシヒカリ、宮城にはひとめぼれ、青森にはつがるロマン。日本が誇るおいしいお米にも、世界のマーケットという大きなチャンスが広がります。\par
アメリカでは最近、とりわけ流行に敏感なニューヨーカーたちの間で霜降りの「wagyu beef」が人気を集めています。\par
しかし、26%もの関税がかかり、価格がどうしても高くなる。大きな壁として立ちはだかってきました。\par
この壁がTPPによって取り払われます。最大で現在の輸出実績の40倍まで関税がゼロとなります。そして、将来的には全ての制限が取り払われます。米国の皆さんに日本のおいしい和牛をもっと知ってもらい、もっと食べてもらう大きなきっかけとなる。私はそう確信しています。\par
政府としてTPPにチャンスを見出し、世界のマーケットに挑戦しようとする皆さんを全力で応援したいと考えています。\par
この20年近く、日本経済はデフレに苦しんでまいりました。\par
頑張っても報われない。収入が増えない。全ては日本の隅々にまで内向きなマインドが蔓延し、私たちが新たな挑戦を恐れてきた。その結果ではないでしょうか。\par
少子高齢化の進展、経済のグローバル化、新興国の台頭。内外の経済情勢は絶えず変化を続けています。\par
改革を恐れるのはもうやめましょう。勇気を持ってチャレンジすべきです。イノベーションを起こし、そして、オープンな世界に踏み出すべきときであります。\par
TPPは、そのスタートにすぎません。その先には、RCEP、さらにはFTAAPと、アジアの国々と共にもっと大きな経済圏をつくり上げていく。ヨーロッパとのEPA交渉も、年内の合意を目指し、加速しなければなりません。日本はこれからもリーダーシップを発揮していく決意であります。\par
70年前、日本は全てを失いました。\par
しかし、アジアでいち早くGATTに加入し、貿易の自由化を進めました。自動車やエレクトロニクスといった新しい産業を果敢に興し、世界の競争へと打って出ました。そして、わずか20年ほどで、アメリカに次ぐ世界第2位の経済大国に上り詰めました。\par
先人たちの血のにじむような努力によって現在の繁栄がある。今を生きる私たちもまた、力の限りを尽くして、日本を更に成長させ、子や孫の世代へと引き渡していく大きな責任があります。\par
その責任を果たすため、国民の皆さんと共に、今日、ここから新たな一歩を踏み出したい。TPPへの参加について、国民の皆様の御理解と御支援をお願いする次第であります。\par
私からは以上であります。\par

\section {質疑応答}
\begin {itemize}
\item (内閣広報官)それでは、皆様からの質問をいただきます。\par
  最初に、幹事社の方から御質問、会社のお名前と記者の方のお名前、明記の上でお願いいたします。どうぞ。\par
\item (記者)幹事社、北海道新聞の小林と申します。\par
  総理は2年前に、国益にかなう最善の道を追求すると国民に約束された上で、TPP交渉への参加を決断されました。今回の合意内容については、経済界から歓迎する声が上がっている一方、先ほどもおっしゃられた米の無関税輸入枠が新設されたことなどから、農業団体などには、農産品の重要5品目など聖域確保を優先し、確保できない場合には交渉脱退も辞さないとした国会決議に反するのではないかというような声も上がっています。総理は、今回の合意内容について、日本が守るべき聖域は守られたとお考えでしょうか。また、TPPによって影響を受ける国内産業に対する対策について、具体的な時期や規模などについてのお考えを教えてください。\par
\item (安倍総理)平成25年4月の衆参の農林水産委員会において、TPP交渉に関し、米、麦、牛肉・豚肉、乳製品、甘味資源作物などの農林水産物の重要品目について、引き続き再生産可能となるよう、除外または再協議の対象とすること。10年を超える期間をかけた段階的な関税撤廃も認めないこと。農林水産分野の重要5品目などの聖域の確保を最優先し、それが確保できないと判断した場合は脱退も辞さないものにすることなどを内容とする決議がなされました。\par
  TPPは包括的で高い水準の協定を目指し、関税撤廃の圧力が極めて強かったわけでありますが、政府としてはこの決議をしっかりと受けとめ、同年7月の正式交渉参加以来、ぎりぎりの交渉を行ってまいりました。その結果、米国等が近年締結しているFTAでは類例を見ないようなレベルで重要5品目を中心に関税撤廃の例外を数多く確保することができました。\par
  さらに、国会決議を後ろ盾に各国と粘り強く交渉し、重要5品目を中心に国家貿易制度を堅持するとともに、既存の関税割当品目の枠外税率を維持したことに加えまして、関税割り当てやセーフガードの創設、関税削減期間を長期とするなどの有効な措置を認めさせることができました。\par
  農業は国の基であり、美しい田園風景を守っていくことは政治の責任であります。農林水産業を意欲ある生産者が安心して再生産に取り組むことができる。若い皆さんにとって夢のある分野にしていく考えであります。\par
  今後、どのような具体的影響が生じ得るかを十分に精査していきます。その上でTPP協定の締結について、国会の承認を求めるまでの間に政府全体で責任を持って国内対策を取りまとめ、交渉で獲得した措置と併せて万全の措置を講じていく考えであります。\par
\item (内閣広報官)それでは、幹事社からもう一問いただきます。どうぞ。\par
\item (記者)幹事社のフジテレビの西垣です。お疲れさまです。\par
  関連しまして、内閣改造についてお伺いします。TPPに関しましては、交渉参加から今回の大筋合意に至りました。また、内閣の方針として先般の平和安全保障法制などもあり、担当大臣を置かれて内閣で取り組まれてきたことが節目をこうして迎えていく中、職務のあり方というものも変わっていかれると思います。\par
  また、次の内閣改造では総理は「1億総活躍社会」というものを掲げていらっしゃいますが、もし、この節目を迎えて内閣の取組の方針というものは今後どう変わるのか。また、それに伴い、どういった方を登用されるのか。お考えについてお伺いさせていただきます。\par
\item (安倍総理)少子高齢化社会に歯止めをかけ、誰もが活躍できる「1億総活躍社会」をつくるのは、その社会づくりは、最初から設計図があるような簡単な課題ではありません。「希望出生率1.8」、「介護離職ゼロ」などの野心的な目標を実現するためには内閣一丸となって、今までの発想にとらわれない大胆な政策を立案し、実行していくことが必要であります。\par
  その司令塔たる1億総活躍担当大臣には、省庁の縦割りを排した広い視野と、そして大胆な政策を構想する発想力。さらには、それを確実に実行する強い突破力が必要であろう、求められると思います。\par
  奇をてらうのではなくて、仕事重視、結果第一の体制。まさに新しい体制においてしっかりと結果を出していくことのできる、そうした内閣にしていきたい。そうした人事を行っていきたいと考えています。\par
\item (内閣広報官)それでは、これから幹事社以外の方からの質問をいただきますので、御希望される方は挙手をお願いいたします。私のほうから指名させていただきますので、その方は改めて所属とお名前を明らかにした上で質問に入らせていただきます。\par
  それでは、リンダさん。\par
\item (記者)ロイターのシーグと申します。\par
  今年の4月にアメリカのカーター国防長官が、TPP協定は空母と同じぐらいの意味があるとおっしゃいました。要するに経済的なメリットだけではなくて、地域に対して非常に戦略的な意義が大きいということですが、総理は、日米関係、日中関係、地域全体にとっての戦略的な意義はどう見ていらっしゃるのでしょうか。特に、TPP協定は、中国に対してどういうメッセージを送るでしょうか。\par
\item (安倍総理)TPPは、アジア・太平洋に自由、民主主義、基本的人権、そして、法の支配といった基本的価値を共有する国々と共に自由で、公正、開かれた国際経済システムをつくり上げ、経済面での法の支配を抜本的に強化するものであります。\par
  新たな時代に適したルールを基にこうした国々と相互依存関係を深めていくことは、そしてそのことこそ、将来的に中国もそのシステムに参加すれば、我が国の安全保障にとっても、また、アジア太平洋地域の安定にも大きく寄与し、戦略的にも非常に大きな意義があると思います。\par
  日本と米国という世界第1位と第3位の経済大国が参加してつくられるTPPは、世界最大の経済圏となります。現在交渉中の日・EU経済連携協定、EPA交渉にも大きな弾みを与えることになるのは間違いないと思います。\par
  TPPによってつくられる新たな経済秩序は、単にTPPだけにとどまらず、その先にある東アジア地域包括的経済連携、RCEPや、もっと大きな構想であるアジア太平洋自由貿易圏、FTAAPにおいて、そのルールづくりのたたき台となり、21世紀の世界のスタンダードになっていくという大きな意義があると思います。\par
\item (内閣広報官)それでは、もう一問だけいただきたいと思います。\par
  では、松本さん。\par
\item (記者)時事通信の松本です。お伺いします。\par
  野党内には、今回のTPP交渉の経緯や情報開示を求める声がございまして、早期の国会審議を求める意見があります。\par
  一方で、政府・与党内には、臨時国会を見送る考え方もあるようですけれども、総理としては、こうした野党の声にどう応えるお考えでしょうか。また、臨時国会の開催についても、どうお考えか、現時点のお考えをお聞かせください。\par
\item (安倍総理)このTPP協定によって、消費者が海外の、より良いものを便利に、より安く手に入れることができるようになります。\par
  同時に、例えば農家の方々が良いものをつくれば、海外で、そして、それが高く評価されれば、今まで輸出できなかった国にも輸出できるようになるわけであります。言ってみれば、付加価値が正しく、高く評価される経済システムとなっていくと思います。\par
  アジアの新興国を中心に、自動車や自動車部品など、鉱工業製品に高い関税が課されていましたが、TPP協定によって、これらの関税のほとんど全て、最終的に撤廃されることになります。\par
  金融や流通など、サービスや投資分野での参入規制が緩和され、地域の金融機関やコンビニなどの海外展開が容易になります。国有企業との公正な競争条件の確保、インフラ市場への参入拡大なども、我が国企業の海外展開の大きな助けとなることが期待されます。\par
  知的財産に関するルールの調和、海賊版、模倣品対策の強化など、日本の強みであるコンテンツや地域ブランドの海外展開が安心して進められるようになります。\par
  そして、また、TPPは、地域の中小・中堅企業に大きなチャンスをもたらします。これは、余り理解されていないかもしれません。大企業にしかチャンスがないのではないかと思われているかもしれませんが、地方、地域の中小・小規模事業者の皆さんにも大きなチャンスをもたらすことになるのは間違いありません。\par
  インターネットによる取引のルールが整備されることで、中小・中堅企業が日本にいながらにして、アジア・太平洋全域にビジネスを展開していくことが可能となります。\par
  原産地のルールが整備されていくことで、中小企業が日本に生産拠点を維持しながら海外でビジネスを展開することが容易になります。\par
  また、TPPがもたらすメリットを中小企業が最大限活用できるよう、情報提供の強化やセミナーの開催等、中小企業によるTPP利用促進のための様々な仕組みがTPP協定に組み込まれることになるわけであります。\par
  TPPは消費者や働く人にもメリットをもたらすわけでありますし、消費者がネット取引を通じて海外から様々な製品を手軽にかつ安心して取り寄せることができるようになり、また働く人々にとっても各国が労働基準や環境基準をしっかりと守るようなルールが盛り込まれたことで、公平な競争条件が確保されたと考えております。\par
  我が国の企業がTPP協定を最大限活用し、TPP協定が真に我が国の経済再生、地方創生につながるよう、万全の施策を講じていきたいと考えております。\par
  こうしたTPPの正しい姿を、どのようなメリットがあるかということもしっかりと我々は説明をしていきたいと、こう考えています。\par
  そして、臨時国会についてでありますが、これは与党ともよく相談しながら考えていきたいと思います。今月も、11月にも多くの国際会議や海外出張が既に予定をされているわけでありますが、党と相談をしながら決めていきたいと思います。\par
  いずれにせよ、このTPPについては国会で批准していただかなければならないわけでありまして、いずれにせよ国会で審議をしていただくことになるわけであります。\par
\item (内閣広報官)予定をしておりました時間も経過いたしましたので、以上をもちまして、安倍総理大臣の記者会見を終わらせていただきます。\par
  皆様、どうも御協力をありがとうございました。\par
\item (安倍総理)ありがとうございました。\par
\end {itemize}

\clearpage
\begin {center}
宮沢経済産業大臣談話・声明(平成27年10月6日)
\end {center}

\section {環太平洋パートナーシップ協定(TPP協定)の大筋合意について}
\begin {enumerate}
\item 10月5日(現地時間)、2013年より交渉を重ねてきたTPP協定の交渉が大筋合意に至った。本協定は、我が国の成長戦略の主要な柱の一つであるとともに、アジア太平洋地域の成長・繁栄・安定に資する重要な枠組みであり、大筋合意に至ったことは大変喜ばしい。

\item 本協定は、世界のGDPの約4割を占める巨大な経済圏において、関税の削減・撤廃だけではなく、投資、サービス、知的財産、国有企業、電子商取引など幅広い分野において21世紀型の貿易投資ルールを構築するものであり、成長著しいアジア太平洋地域に大きなバリュー・チェーンを作り出すことができる。

\item 関税削減・撤廃や、輸出手続きの簡素化、電子商取引をはじめとする本協定上の諸ルールは、我が国の中小・中堅企業にも大きなメリットがある。それに加えて、本協定では中小企業がTPPを利活用するにあたって有益な情報を提供するウェブサイトの設置など、中小企業のビジネス環境を整備するための規律にも合意した。

\item 工業製品については、我が国自身は、過去のWTO交渉等の結果として輸入額の大部分で、既に高いレベルの自由化を達成しているところである。このような中、本協定を通じて、我が国から参加11か国への輸出額(約19兆円)の99.9%についての関税が撤廃されることとなり、我が国からの輸出や国内投資の拡大に大きく貢献することが期待される。

\item 特に、我が国がこれまでEPAを締結していなかった米国、カナダ、ニュージーランドの3か国との間では、20年ぶりの市場アクセスの大幅な改善となる。具体的には、我が国からこれら3か国向けの工業製品輸出額(約12兆円)の約7割の輸出についての関税が発効と同時に撤廃され、最終的に全ての関税が撤廃されることになる。

\item 関税分野以外でも、投資、サービス、知的財産、国有企業、電子商取引など幅広い分野において、日本企業の海外での事業活動の推進に資するとともに、今後の貿易・投資ルールの新たなスタンダードともなる高い水準の合意がなされた。

\item 例えば、ベトナムやマレーシアでの小売業の外資規制緩和などにより日本企業の市場参入機会の拡大が期待されるほか、技術移転要求の禁止やライセンス契約の対価(ロイヤリティ)への政府介入の禁止、各国での商標権の取得手続きの円滑化や模倣品・海賊版への取り締まり強化等を通じて日本企業の海外での投資・知的財産への保護の強化が実現される。

\item 中小・中堅企業を含む我が国企業が本協定を最大限活用することで、大きなビジネスチャンスを掴んでいけるよう万全の施策を講じる。本協定の一日も早い署名に向けて最大限の努力をしていくとともに、TPP協定が我が国の経済再生・地方創生に役立つものとなるよう、全力を尽くしてまいりたい。
\end {enumerate}

\end{document}

