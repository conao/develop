\documentstyle[a4,11pt,epsfig]{jarticle}
\topmargin=-1.5cm
\textwidth=17cm
\textheight=25cm
\oddsidemargin=-0.5cm
\evensidemargin=-0.5cm

\begin{document}
\begin{center}
  \Large\bf 情報活用演習 課題10
\end{center}

\begin{flushright}
  \begin{tabular}{ll}
    学生番号: & B151235 \\
    氏名:     & 山下 直哉 \\
    提出日:   & 平成27年12月15日 \\
    提出期限: & 平成27年12月22日 \\
  \end{tabular}
\end{flushright}

\section {確率分布の例}
代表的な確率分布を示す。
\begin{itemize}
\item 二項分布\\
  $0,1,2,\ldots,n$のいずれかの値をとる離散型確率変数$X$について
  \begin {eqnarray}
    P(X=x)=_nC_xp^xq^{n-x}\ \ (x=0,1,2,\ldots,n; 0 < p <1, p+q = 1)
  \end {eqnarray}
  と表せる確率分布を二項分布と言い、$Bin(n,p)$と書く。

\item ポアソン分布\\
  $0,1,2,\ldots,n$のいずれかの値をとる離散型確率変数$X$について
  \begin {eqnarray}
    P(X=x)=e^{-\lambda}\frac{\lambda^x}{x!}\ \ (x=0,1,2,\ldots,n; \lambda>0)
  \end {eqnarray}
  と表せる確率分布をポアソン分布と言い、$Po(\lambda)$と書く。

\item 正規分布\\
  連続型確率変数$X$が密度関数
  \begin {eqnarray}
    f(x) = \frac{1}{\sqrt{2\pi}\sigma} e^{\frac{(x-\mu)^2}{2\sigma^2}}
  \end {eqnarray}
  にしたがう分布を正規分布と言い$N(\mu,\sigma^2)$で表す。

\item 指数分布\\
  連続型確率変数$X$の分布が密度関数
  \begin {eqnarray}
    f(x) = \left\{
    \begin {array}{ll}
      \lambda e^{-\lambda x}\ \ \ &(x\geq 0)\ (\lambda: 正定数)\\
      0      & (x < 0)
    \end {array}
    \right.
  \end {eqnarray}
  にしたがうとき、指数分布という

\item 一様分布\\
  連続型確率変数$X$の分布が密度関数
  \begin {eqnarray}
    f(x) = \left\{
    \begin {array}{ll}
      \frac{1}{n}\ \ \ & (x=x_1,x_2,\ldots,x_n)\\
      0      & (その他のx)
    \end {array}
    \right.
  \end {eqnarray}
  に従うとき、一様分布という。
\end {itemize}

\end{document}

