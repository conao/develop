\documentstyle[a4,11pt]{jarticle}
\topmargin=-1.5cm
\textwidth=17cm
\textheight=25cm
\oddsidemargin=-0.5cm
\evensidemargin=-0.5cm

\begin{document}
\begin{center}
  \Large\bf 情報活用演習 課題5
\end{center}

\begin{flushright}
  \begin{tabular}{ll}
    学生番号: & B151235 \\
    氏名:     & 山下 直哉 \\
    提出日:   & 平成27年11月10日 \\
    提出期限: & 平成27年11月17日 \\
  \end{tabular}
\end{flushright}

\begin{center}
  内閣総理大臣談話\\
\end{center}

\section {内閣総理冒頭発言}
止まらぬデフレ。美しい海や国土に迫る脅威。3年前、国家的な危機に直面していた日本は、「黄昏を迎えている」とまで言われていました。その危機感が、政権交代の大きな原動力となりました。\par
あれから1000日余り。「三本の矢」の経済政策、そして戦後以来の大改革。「日本を取り戻す」ため、安倍内閣は、全力を尽くしてまいりました。\par
雇用は100万人以上増加し、有効求人倍率は23年ぶりの高さです。2年連続で2%を上回る賃上げが実現し、この春は、17年ぶりの高い伸びとなりました。中小・小規模事業者の倒産件数も、大きく減少し、この24年間で最低の水準です。\par
アベノミクスの1000日は、経済の好循環を生み出し、もはやデフレではない、という状態を創ることができました。デフレ脱却は、もう目の前です。\par
東日本大震災の被災地では、災害公営住宅への入居が進み、住まいと生業の復興が、着実に進んでいます。\par
日米同盟の絆は完全に復活し、地球儀を俯瞰する視点で進めてきた、積極的な経済外交、平和外交は、大きな実を結びつつあります。安全保障体制の基盤は、一層強化されました。\par
日本は、ようやく「新しい朝」を迎えることができました。\par
今こそ、新たな国づくりを始めなければなりません。\par
少子高齢化の流れに歯止めをかけ、50年後も人口一億人を維持する。そして、高齢者も若者も、女性も男性も、難病や障害を抱える人も、誰もが、今よりももう一歩前へ、踏み出すことができる社会を創り上げる。\par
その力強いスタートを切るため、本日、内閣を改造いたしました。\par
新たな陣容のもと、「一億総活躍」という旗を高く掲げ、「戦後最大のGDP600兆円」、「希望出生率1.8」、「介護離職ゼロ」という、3つの大きな目標に向かって、新たな「三本の矢」を力強く放ちます。\par
いずれも、長年の懸案であった構造的課題へのチャレンジです。最初から設計図が用意されているような、簡単な目標ではありません。\par
しかし、目を背けてはならない。諦めてはいけません。私たちの子や孫の世代に「誇りある日本」を引き渡す。これは、今を生きる私たちの、大きな責任であります。\par
強い日本。それを創るのは、他の誰でもありません。私たち自身です。\par
国民の皆さんと共に、この大きな課題に果敢に挑む。私は、その先頭にあって、全身全霊を傾けていく覚悟であります。次なる安倍内閣のチャレンジに、国民の皆さんの御理解と御協力を、改めて、お願いする次第です。\par
\clearpage

\begin {center}
  法律の原案作成から法律の公布まで\\
\end {center}

\subsection {法律案の原案作成}
\begin {itemize}
\item 内閣が提出する法律案の原案の作成は、それを所管する各省庁において行われます。\par
  各省庁は所管行政の遂行上決定された施策目標を実現するため、新たな法律の制定又は既存の法律の改正若しくは廃止の方針が決定されると、法律案の第一次案を作成します。\par

\item この第一次案を基に関係する省庁との意見調整等が行われます。\par
  更に、審議会に対する諮問又は公聴会における意見聴取等を必要とする場合には、これらの手続を済ませます。\par
  そして、法律案提出の見通しがつくと、その主管省庁は、法文化の作業を行い、法律案の原案ができ上がります。\par
\end {itemize}

\subsection {内閣法制局における審査}
\begin {itemize}
\item 内閣が提出する法律案については、閣議に付される前にすべて内閣法制局における審査が行われます。\par
  内閣法制局における審査は、本来、その法律案に係る主管省庁から出された内閣総理大臣あての閣議請議案の送付を受けてから開始されるものでありますが、現在、事務的には主管省庁の議がまとまった法律案の原案について、いわば予備審査の形で進める方法が採られています。\par
  したがって、閣議請議は、内閣法制局の予備審査を経た法律案に基づいて行われます。\par
\item 内閣法制局における審査は、主管省庁で立案した原案に対して、\par
  \begin {itemize}
  \item 憲法や他の現行の法制との関係、立法内容の法的妥当性、\par
  \item 立案の意図が、法文の上に正確に表現されているか、\par
  \item 条文の表現及び配列等の構成は適当であるか、\par
  \item 用字・用語について誤りはないか\par
  \end {itemize}
  というような点について、法律的、立法技術的にあらゆる角度から検討します。\par
\item 予備審査が一応終了すると主任の国務大臣から内閣総理大臣に対し国会提出について閣議請議の手続を行うことになり、これを受け付けた内閣官房から内閣法制局に対し同請議案が送付されますが、内閣法制局では、予備審査における審査の結果とも照らし合わせつつ、最終的な審査を行い、必要があれば修正の上、内閣官房に回付します。\par
\end {itemize}

\subsection {国会提出のための閣議決定}
\begin {itemize}
\item 閣議請議された法律案については、異議なく閣議決定が行われると、内閣総理大臣からその法律案が国会(衆議院又は参議院)に提出されます。\par
  なお、内閣提出法律案の国会提出に係る事務は、内閣官房が行っています。\par
\end {itemize}

\subsection {国会における審議}
\begin {itemize}
\item 内閣提出の法律案が衆議院又は参議院に提出されると、原則として、その法律案の提出を受けた議院の議長は、これを適当な委員会に付託します。\par
  委員会における審議は、まず、国務大臣の法律案の提案理由説明から始まり、審査に入ります。審査は、主として法律案に対する質疑応答の形式で進められます。\par
  委員会における質疑、討論が終局したときは、委員長が、問題を宣告して、表決に付します。\par
  委員会における法律案の審議が終了すれば、その審議は、本会議に移行します。\par
\item 内閣提出の法律案が、衆議院又は参議院のいずれか先に提出された議院において、委員会及び本会議の表決の手続を経て可決されると、その法律案は、他の議院に送付されます。\par
  送付を受けた議院においても、委員会及び本会議の審議、表決の手続が行われます。\par
\end {itemize}

\subsection {法律の成立}
\begin {itemize}
\item 法律案は、憲法に特別の定めのある場合を除いては、衆議院及び参議院の両議院で可決したとき法律となります。\par
  こうして、法律が成立したときは、後議院の議長から内閣を経由して奏上されます。\par
\end {itemize}

\subsection {法律の公布}
\begin {itemize}
\item 法律は、法律の成立後、後議院の議長から内閣を経由して奏上された日から30日以内に公布されなければなりません。\par
  法律の公布に当たっては、公布のための閣議決定を経た上、官報に掲載されることによって行われます。\par
  (官報では、公布された法律について、一般の理解に資するため「法令のあらまし」が掲載されています。)\par
\item 「公布」は、成立した法律を一般に周知させる目的で、国民が知ることのできる状態に置くことをいい、法律が現実に発効し、作用するためには、それが公布されることが必要です。\par
  なお、法律の効力が一般的、現実的に発動し、作用することになることを「施行」といい、公布された法律がいつから施行されるかについては、通常、その法律の附則で定められています。\par
\item 法律の公布に当たっては、その法律に法律番号が付けられ、主任の国務大臣の署名及び内閣総理大臣の連署がされます。\par
\end {itemize}
\end{document}

