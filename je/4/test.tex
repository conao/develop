\documentstyle[a4]{jarticle}
\textwidth=18cm
\oddsidemargin=-0.5cm
\textheight=26cm
\topmargin=-1.5cm
\begin{document}
\begin{center}
  \large\bf 課題4 解答
\end{center}
\begin{flushright}
  \begin{tabular}{ll}
    学生番号 & B151235\\
    氏名 & 山下 直哉\\
  \end{tabular}
\end{flushright}
\begin{itemize}
\item コモディティ化
  \begin{itemize}
  \item 意味\\
    \begin{quote}
      ある商品カテゴリにおいて、競争商品間の差別化特性(機能、品質、ブランド力など)が失われ、主に価格あるいは量を判断基準に売買が行われるようになること。一般に商品価格の下落を招くことが多く、高価な商品が低価格化・普及品化することを“コモディティ化”という場合もある。\\
    \end{quote}
  \item 出典\\
    \begin{quote}
      情報システム用語事典:コモディティ化(こもでぃてぃか)commoditization\\
      http://www.itmedia.co.jp/im/articles/0505/28/news013.html\\
    \end{quote}
  \item 利用例・応用例\\
    多くの会社で製造可能となるり、機能や品質の面で差のない製品が市場に多数投入されることにより、顧客は価格(コスト)あるいは買いやすさ(店頭にあるかなど)以外に選択要因がなくなる。こうした状態のことをコモディティ化という。いい換えれば、「どの会社のものを買っても同じ」という状態のこと。
  \end{itemize}
\item ITS
  \begin{itemize}
  \item 意味\\
    \begin{quote}
      ITS(Intelligent Transport Systems:高度道路交通システム)とは、人と道路と自動車の間で情報の受発信を行い、道路交通が抱える事故や渋滞、環境対策など、様々な課題を解決するためのシステムとして考えられました。常に最先端の情報通信や制御技術を活用して、道路交通の最適化を図ると同時に、事故や渋滞の解消、省エネや環境との共存を図っていきます。関連技術は多岐にわたり、社会システムを大きく変えるプロジェクトとして、新しい産業や市場を作り出す可能性を秘めています。\\
    \end{quote}
  \item 出典\\
    \begin{qupte}
      特定非営利活動法人 ITS Japan - ITSとは\\
      http://www.its-jp.org/about/\\
    \end{end}
  \item 利用例・応用例\\
    \begin{quote}
      日本においては、以下の9つの開発分野に分けるシステム分類がある。\\
      \\
      ナビゲーションシステムの高度化\\
      VICS\\
      自動料金収受システム\\
      ETC\\
      安全運転の支援\\
      AHS(高速道路を中心とした安全運転の支援システム)\\
      DSSS(一般道路を中心とした安全運転の支援システム)\\
      先進安全自動車(ASV、車両を中心とした安全運転の支援システム)\\
      交通管理の最適化\\
      UTMS(交通信号機を核とする警察版のITS)\\
      駐車場案内システム\\
      道路管理の効率化\\
      公共交通の支援\\
      PTPS(公共車両優先システム)\\
      TDM(交通需要マネジメント)\\
      IMTS (磁気誘導式鉄道とも呼ぶ。法的には鉄道扱い)\\
      デマンドバス\\
      パークアンドライド\\
      商用車の効率化\\
      共同配送\\
      ロケーション管理システム\\
      歩行者等の支援\\
      緊急車両の運行支援\\
    \end{quote}
  \end{itemize}
\end{itemize}
\end{document}
