\documentstyle[a4,11pt]{jarticle}
\topmargin=-1.5cm
\textwidth=17cm
\textheight=25cm
\oddsidemargin=-0.5cm
\evensidemargin=-0.5cm

\begin{document}
\begin{center}
  \Large\bf 情報活用演習課題7
\end{center}

\begin{flushright}
  \begin{tabular}{ll}
    学生番号: & xxxx \\
    氏名:     & xxxx \\
    提出日:   & 平成○年○月○日 \\
    提出期限: & 平成○年○月○日 \\
  \end{tabular}
\end{flushright}

\begin{center}
  \large\bf 
  平成26年度情報通信白書
\end{center}

\section{用語解説}


\begin{table}[htb]
  \begin{center}
    \caption{用語1}
    \begin{tabular}{|c|c|p{12cm}|}
      \hline
      索引 & 用語 & \multicolumn{1}{c|}{用語解説} \\
      \hline
      A &  ARPU & Average Revenue Per Userの略。加入者一人当たりの平均利用月額。\\
      \cline{2-3}
      & ASPIC & ASP-SaaS-Cloud Consortiumの略。特定非営利活動法人 ASP・ SaaS・クラウ
      ド コンソーシアム。クラウド・ASP・SaaS・データセンター事業の発展と支援を目的とし
      て、1999年に設立された。 
      \\
      \cline{2-3}
      & API & Application Programming Interfaceの略で、アプリケーションの開発者が、他
      のハードウェアやソフトウェアの提供している機能を利用するためのプログラム上の手続
      きを定めた規約の集合を指す。個々の開発者は規約に従ってその機能を「呼び出す」だけ
      で、自分でプログラミングすることなくその機能を利用したアプリケーションを作成する
      ことができる。 \\
      \cline{2-3}
      & ASP & Application Service Providerの略。ビジネス用アプリケーションソフトをイン
      ターネットを通じて顧客に提供する事業者。 \\
      \hline
      B & BWA & Broadband Wireless Accessの略。信号を伝えるケーブルの代わりに無線(電
        波)を使うデータ通信サービスの総称。無線アクセスシステム。 \\
      \cline{2-3}
      & BS放送 & 静止衛星を用いて行われる放送のうち、放送専用の衛星( Broadcasting
        Satellite)を用いるもの。なお、通信衛星(Communication Satellite)を用いて行われ
      る放送は CS放送。\\ 
      \cline{2-3}
      & BCP & Business Continuity Planの略。何らかの障害が発生した場合に重要な業務が中
      断しないこと、または業務が中断した場合でも目標とした復旧時間内に事業が再開できる
      ようにするための対応策などを定めた包括的な行動計画。 \\
      \cline{2-3}
      & Bluetooth & 無線 LANのようにデータの送受信を行うための無線通信の規格。最大通信
      距離が無線 LANより短い半面、消費電力が少ないという利点があり、ワイヤレスイヤホン
      等の機器に使用される。 \\
      \hline
      C &  CIO & Chief Information Officerの略。日本語では「最高情報責任者」「情報シス
        テム担当役員」「情報戦略統括役員」など。企業や行政機関等といった組織において情報
      化戦略を立案、実行する責任者のこと。 \\
      \cline{2-3}
      & CS放送 & → BS放送の項を参照。\\
      \cline{2-3}
      & CSV & Comma-Separated Valuesの略。いくつかのフィールド(項目)をカンマ「 ,」で
      区切ったテキストデータ及びテキストファイル。長らく公式な仕様が存在しなかったが、
      2005年 10月に RFC4180として規格化された。 \\
      \hline
    \end{tabular}
  \end{center}
\end{table}

\begin{table}[htb]
  \begin{center}
    \caption{用語2}
    \begin{tabular}{|c|c|p{12cm}|}
      \hline
      索引 & 用語 & \multicolumn{1}{c|}{用語解説} \\
      \hline
      D & DSL & Digital Subscriber Lineの略。デジタル加入者回線。電話用のメタリックケ
      ーブルにモデム等を設置することにより、高速のデジタルデータ伝送を可能とする方式の
      総称。\\
      \cline{2-3}
      & DoS攻撃 & DoSは Denial of Serviceの略。サービス妨害攻撃。標的となるコンピュー
      タやルータに大量のデータを送りつけることにより、当該宛先のシステムを動作不能とす
      る攻撃。\\
      \hline
      E & EHR & Electronic Health Recordの略。電子健康記録。個人が自らの健康情報(診療
        情報、レセプト情報、健診結果情報及び健康関連情報)を電子的に活用しようとするもの
      。\\
      \hline
      F & FTTH & Fiber To The Homeの略。各家庭まで光ファイバケーブルを敷設することによ
      り、数十〜最大 1Gbps程度の超高速インターネットアクセスが可能。\\
      \cline{2-3}
      & FWA & Fixed Wireless Accessの略。加入者系無線アクセスシステム。P-P(対向)方式
      、P-MP(1対多)方式があり、それぞれ最大百数十Mbps、10Mbpsの通信が実現可能。\\
      \hline
      G & GDP & Gross Domestic Productの略。国民総生産(GNP)から海外で得た純所得を差
      し引いたもので、国内の経済活動の水準を表す指標となる。\\
      \cline{2-3}
      & GC接続 & Group unit Center(加入者交換局)接続の略。 NTT東日本・ NTT西日本地域
      会社以外の電気通信事業者が、NTT東日本・西日本のネットワークと加入者交換局レベル
      で相互接続することを指す。\\
      \cline{2-3}
      & GPS& Global Positioning Systemの略。全地球測位システム。人工衛星を利用して、利
      用者の地球上における現在位置を正確に把握するシステム。 \\
      \cline{2-3}
      & G空間情報 & 地理空間情報と同義。地理空間情報は地理空間情報活用推進基本法(平成
        19年法律第 63号)において、位置情報(空間上の特定の地点又は区域の位置を示す情報
        (当該情報に係る時点に関する情報を含む。))及び位置情報に関連付けられた情報と定
      義。 \\
      \hline
      H & HTML & HyperText Markup Languageの略。 WWWコンソーシアムが策定している規格の
      一つでウェブページを記述するためのマークアップ言語。 \\
      \cline{2-3}
      & HTML5 & 現在、 WWWコンソーシアムで改訂作業が行われている HTMLの規格。 2014年(
        平成 26年)に勧告化される予定。 \\
      \cline{2-3}
      & HEMS & Home Energy Management Systemの略。家庭用エネルギー管理システム。住宅に
      ICT技術を活用したネットワーク対応型の省エネマネジメント装置を設置し、自動制御に
      よる省エネルギー対策を推進するシステム。 \\
      \hline
    \end{tabular}
  \end{center}
\end{table}

\begin{table}[htb]
  \begin{center}
    \caption{用語3}
    \begin{tabular}{|c|c|p{10cm}|}
      \hline
      索引 & 用語 & \multicolumn{1}{c|}{用語解説} \\
      \hline
      I &  ICT & Information \& Communications Technologyの略。
      ISPInternet Services Providerの略。インターネット接続業者。電話回線や ISDN回線、
      ADSL回線、光ファイバー回線、データ通信専用回線などを通じて、コンピュータをインタ
      ーネットに接続する。 \\
      \cline{2-3}
      & IPv6 & Internet Protocol version 6の略。現在広く使用されているインターネットプ
      ロトコル(IPv4)の次期規格であり、 IPv4に比べて、アドレス数の大幅な増加、セキュ
      リティの強化及び各種設定の簡素化等が実現可能。 \\
      \cline{2-3}
      & IT戦略本部 & 
      高度情報通信ネットワーク社会推進戦略本部。 ITの活用により世界的規模で生じている
      急激かつ大幅な社会経済構造の変化に適確に対応することの緊要性にかんがみ、高度情報
      通信ネットワーク社会の形成に関する施策を迅速かつ重点的に推進するために、平成 13
      年 1月、内閣に設置された。\\
      \cline{2-3}
      & ITS & Intelligent Transport Systemsの略。高度道路交通システム。情報通信技術等
      を活用し、人と道路と車両を一体のシステムとして構築することで、渋滞、交通事故、環
      境悪化等の道路交通問題の解決を図るもの。 \\
      \cline{2-3}
      & IC接続 & 
      Intra-zone Center(中継交換局)接続の略。NTT東日本・NTT西日本地域会社以外の電気
      通信事業者が、 NTT東日本・西日本のネットワークと中継交換局レベルで相互接続するこ
      と。中継交換局とは、GCからの回線を集約し、他局に中継している局のこと。ZC(Zone
        Center)接続ともいう。\\
      \cline{2-3}
      & IP-VPN & Internet Protocol-Virtual Ptivate Networkの略。電気通信事業者の閉域
      IP通信網を経由して構築された仮想私設通信網。 IP-VPNを利用することにより、遠隔地
      のネットワーク同士を LAN同様に運用することが可能。 \\
      \cline{2-3}
      & IP電話 & 通信ネットワークの一部又は全部において IP(インターネットプロトコル)
      技術を利用して提供する音声電話サービス。 \\
      \cline{2-3}
      & IX & Internet eXchangeの略。インターネット・サービス・プロバイダ(ISP)相互間
      を接続する接続点。この相互接続により、異なるプロバイダに接続しているコンピュータ
      同士の通信が可能。 \\
      \cline{2-3}
      & IPマルチキャスト & 
      IPネットワーク上で、複数の相手を一括指定して同じデータを配信する方式で、単一の相
      手を個別に指定する通常の方式に比べ、効率良くデータを配信することができる。 IPTV
      において多チャンネル放送を実現する際などに用いられる。 \\
      \cline{2-3}
      & IPTV & 放送番組等の映像コンテンツを IPネットワークを通じて配信するサービス。
      \\
      \cline{2-3}
      & Internet of Things(IoT)& モノのインターネット。PCやスマートフォンに限らず、
      センサー、家電、車など様々なモノがインターネットで繋がること。 \\
      \hline
    \end{tabular}
  \end{center}
\end{table}


\begin{table}[htb]
  \begin{center}
    \caption{用語4}
    \begin{tabular}{|c|c|p{12cm}|}
      \hline
      索引 & 用語 & \multicolumn{1}{c|}{用語解説} \\
      \hline
      L & LTE & Long Term Evolutionの略。「3.9G」と呼ばれ、W-CDMAや HSPA規格の後継とな
      る高速データ通信を実現する移動体通信の規格のこと。 
      (関連項目⇒「3.9世代移動通信システム」の項を参照)\\
      \cline{2-3}
      & LAN & Local Area Networkの略。企業内、ビル内、事業所内等の狭い空間においてコン
      ピュータやプリンタ等の機器を接続するネットワーク。 \\
      \hline
      M & M2M & Machine-to-Machineの略。ネットワークに繋がれた機械同士が人間を介在せず
      に相互に情報交換し、自動的に最適な制御が行われるシステムのこと。 \\
      \hline
      N & NCC & New Common Carrierの略。 1985年の通信自由化により新規参入した第一種電
      気通信事業者の総称。新電電とも呼ばれる。主に国内の市外通話を提供している。自由化
      直後は、京セラなどをを母体とする第二電電( DDI)、 JRなどを母体とする日本テレコ
      ム( JT)、日本道路公団などを母体とする日本高速通信( TWJ)の 3社を指していた(
        その後 TWJは KDDに吸収された)。 2000年に DDIとKDDは合併し、KDDIとなった。 \\
      \cline{2-3}
      & NPO & Nonprofit Organizationの略。非営利団体一般のことを指す場合と、特定非営利
      活動促進法により法人格を得た特定非営利活動法人のみを指す場合がある。 \\
      \hline
      O &  
      OS & Operating Systemの略。「基本ソフトウェア」とも呼ばれ、キーボード入力や画面
      出力等の入出力機能、ディスクやメモリの管理など、多くのアプリケーションソフトが共
      通して利用する基本的な機能を提供し、コンピュータシステム全体を管理するソフトウェ
      ア。 \\
      \cline{2-3}
      & OTT & Over The Topの略。自社では通信ネットワークは持たずにコンテンツ等を配信す
      る上位産業レイヤーを指し、代表的なものに SNSやスマートフォンアプリ等の事業者が含
      まれる。 \\
      \hline
      P & PDA & Personal Digital Assistantsの略。個人向けの携帯情報端末であり、パソコ
      ンのもつ機能のうちいくつかを備えている。 \\
      \cline{2-3}
      & POS & Point Of Sales(販売時点管理)システムの略。小売業において個々の店舗にお
      いて商品の販売情報を記録し、これを集計した結果を在庫管理やマーケティングのための
      データとして利活用するシステムのこと。 \\
      \hline
      S &  SNS & Social Networking Service(Site)の略。インターネット上で友人を紹介し
      あって、個人間の交流を支援するサービス(サイト)。誰でも参加できるものと、友人か
      らの紹介がないと参加できないものがある。会員は自身のプロフィール、日記、知人・友
      人関係等を、ネット全体、会員全体、特定のグループ、コミュニティ等を選択の上公開で
      きるほか、 SNS上での知人・友人等の日記、投稿等を閲覧したり、コメントしたり、メッ
      セージを送ったりすることができる。プラグイン等の技術により情報共有や交流を促進す
      る機能を提供したり、 API公開により連携するアプリケーション開発を可能にしたものも
      ある。 \\
      \hline
    \end{tabular}
  \end{center}
\end{table}


\begin{table}[htb]
  \begin{center}
    \caption{用語5}
    \begin{tabular}{|c|c|p{10cm}|}
      \hline
      索引 & 用語 & \multicolumn{1}{c|}{用語解説} \\
      \hline
      & SaaS & Software as a Serviceの略。ネットワークを通じて、アプリケーションソフト
      の機能を顧客の必要に応じて提供する仕組み。 \\
      \cline{2-3}
      & SMS & Short Message Serviceの略。携帯電話同士で短い文字メッセージ又はその他の
      情報を送受信できるサービス。 \\
      \hline
      T & TFP & Total Factor Productivityの略。全要素生産性または総要素生産性。経済成
      長を論じる手法の一つであり、技術進歩による経済生産増への寄与度としてよく使われる
      。\\
      \cline{2-3}
      & Twitter &個々のユーザーが「ツイート」(tweet)と呼ばれる 140文字以内の「つぶや
        き」を投稿し、そのユーザーをフォローしているユーザーが閲覧できるサービス。タイム
      ラインと呼ばれる自分のページには自分の投稿と自分がフォローしているユーザーの投稿
      が時系列順に表示される。 RTによる他人のツイートの引用、ハッシュタグによる特定の
      テーマでのやり取り等の仕組みも取り入れられ、 APIの公開により、様々なサービスが開
      発されている。 \\
      \hline
      W & Wi-Fi無線 & LANの標準規格である「IEEE 802.11a/b/g/n」の消費者への認知を深め
      るため、業界団体のWECA(現:Wi-Fi Alliance)が名付けたブランド名。 \\
      \hline
      X & 
      \begin{tabular}[t]{c}
        XML \\ 
        【eXtensible Markup \\
          Language】\\
      \end{tabular}
      & 
      HTMLと同様に、ウェブぺージを記述する際などに用いる言語であり、テキスト中にタグと
      呼ばれる書式属性を定義する文字列を埋め込み、文字列の位置付け等を記述する。 HTML
      との違いは拡張性にあり、XMLでは任意のタグを定義して HTMLにはない書式属性を定義す
      ることが可能。\\
      \hline
    \end{tabular}
  \end{center}
\end{table}



\begin{table}[htb]
  \begin{center}
    \caption{用語6}
    \begin{tabular}{|c|c|p{10cm}|}
      \hline
      索引 & 用語 & \multicolumn{1}{c|}{用語解説} \\
      \hline
      あ & アプリ & アプリケーションの略
      →アプリケーションの項を参照 \\
      \cline{2-3}
      & アプリケーション & ワープロ・ソフト、表計算ソフト、画像編集ソフトなど、作業の
      目的に応じて使うソフトウェア。 \\
      \cline{2-3}
      & アクセシビリティ & 情報やサービス、ソフトウェア等が、どの程度広汎な人に利用可
      能であるかをあらわす語。特に、高齢者や障害者等、ハンディを持つ人にとって、どの程
      度利用しやすいかということを意味する。 \\
      \cline{2-3}
      & アーカイブ & 文書や記録等を収集、組織化、蓄積・保存すること。 \\
      \cline{2-3}
      & 暗号技術 & 文書や画像等のデータを通信及び保管する際に、第三者による情報の窃取
      を防ぐことを目的として、規定された手順に従いデータを変換し、秘匿化する技術。 \\
      \hline
      い & イノベーション & 新技術の発明や新規のアイデア等から、新しい価値を創造し、社
      会的変化をもたらす自発的な人・組織・社会での幅広い変革のこと。\\
      \cline{2-3}
      & インターフェース & 機器や装置等が他の機器や装置等と交信し、制御を行う接続部分
      のこと。 \\
      \hline
      う & ウイルス & コンピュータシステムの破壊等を目的としたプログラムのこと。電子フ
      ァイル、電子メール等を介して他のファイルに感染することにより、その機能を発揮する
      。 \\
      \hline
      え & 遠隔医療 & 医師と医師、医師と患者との間を ICT(インターネット、テレビ電話な
        ど)を活用して、患者の情報や放射線画像などを伝送し、診断等を行うこと。 \\
      \hline
      お & オープンデータ & 政府が統計・行政などのデータをオープンにすること。Data.gov
      (米国)やData.gov.uk(英国)などの取組が各国政府によって、行われている。 \\
      \cline{2-3}
      & オフロード & 他のシステムに処理を分けることで、あるシステムに対する負荷を軽減
      させる仕組の 1つ。データオフロード等。 \\
      \cline{2-3}
      & オンラインゲーム & インターネットを通して、複数のユーザーが同時に参加すること
      により行われるコンピュータゲーム。 \\
      \hline
      か & 架空請求メール & 架空の料金請求書を無作為にメールで送りつけ、支払いを要求す
      る手口の詐欺、あるいはそのような内容の書かれたメールのこと。\\
      \hline
      く & クラウドコンピューティング & 
      データサービスやインターネット技術等が、ネットワーク上にあるサーバー群(クラウド
        (雲))にあり、ユーザーは今までのように自分のコンピュータでデータを加工・保存す
      ることなく、「どこからでも、必要な時に、必要な機能だけ」利用することができる新し
      いコンピュータ・ネットワークの利用形態。 \\
      \hline
    \end{tabular}
  \end{center}
\end{table}



\begin{table}[htb]
  \begin{center}
    \caption{用語7}
    \begin{tabular}{|c|c|p{10cm}|}
      \hline
      索引 & 用語 & \multicolumn{1}{c|}{用語解説} \\
      \hline
      & クラウドサービス & 
      インターネット等のブロードバンド回線を経由して、データセンタに蓄積されたコンピュ
      ータ資源を役務(サービス)として、第三者(利用者)に対して遠隔地から提供するもの
      。なお、利用者は役務として提供されるコンピュータ資源がいずれの場所に存在している
      か認知できない場合がある。\\
      \hline
      け & ケーブルテレビ & テレビの有線放送サービスのことである。山間部や離島等の難視
      聴地域へ向けて行うために開発された。通信ケーブルが各家庭まで敷設されており、多チ
      ャンネル・双方向のテレビ放送を行うシステムである。 \\
      \hline
      こ & 
      公衆無線 & LAN店舗や公共の空間などで提供される、無線 LANによるインターネット接続
      サービス。(関連項目⇒「無線LAN」の項を参照)\\
      \cline{2-3}
      & コンテンツ & 文字・画像・動画・音声・ゲーム等の情報全般、またはその情報内容の
      こと。電子媒体やネットワークを通じてやり取りされる情報を指して使われる場合が多い
      。 \\
      \cline{2-3}
      & 国内生産額 & 日本国内における生産活動により生産された、製品の生産高やサービス
      の売上高を積み上げたもの。 \\
      \cline{2-3}
      & コンプライアンス & 法令遵守。企業が経営・活動を行ううえで、法令や各種規則など
      のルール、さらには社会的規範などを守ること。 \\
      \cline{2-3}
      & コモディティ化 & ある製品ないし商品の普及が一巡すると、競合製品への優位性が機
      能や品質ではなく主に価格に起因するようになり、その結果、価格低下に拍車がかかる現
      象のこと。 \\
      \cline{2-3}
      & コミュニティ放送 & 市町村単位を放送エリアとする FM放送。放送エリアが小さく、よ
      り地域に密着した番組を放送していることが特徴。 \\
      \cline{2-3}
      & 広域イーサネット & 通信事業者の提供するイーサネット網を利用し、離れた場所にあ
      る複数の LANを接続した大型ネットワーク。 \\
      \hline
      さ & サプライチェーン & 取引先との間の受発注、資材の調達から在庫管理、製品の配達
      まで、いわば事業活動の川上から川下に至るまでのモノ、情報の流れ。 \\
      \cline{2-3}
      & サーバー & ネットワーク上でサービスや情報を提供するコンピュータ。インターネッ
      トではウェブサーバー、DNSサーバー、メールサーバー等があり、ネットワークで発生す
      る様々な業務を、内容に応じて分担し、集中的に処理する。 \\
      \cline{2-3}
      & 3.9世代移動通信システム & 第 3世代移動通信システム(IMT-2000規格)の高度化シス
      テム(3.9G)。3.9世代携帯電話。光ファイバ並みの高速伝送が可能となる。\\
      \hline
    \end{tabular}
  \end{center}
\end{table}


\begin{table}[htb]
  \begin{center}
    \caption{用語8}
    \begin{tabular}{|c|c|p{10cm}|}
      \hline
      索引 & 用語 & \multicolumn{1}{c|}{用語解説} \\
      \hline
      し & 情報セキュリティ & 情報資産を安全に管理し、適切に利用できるように運営する経
      営管理のこと。適切な管理・運営のためには、情報の機密性・安全性・可用性が保たれて
      いることが必要となる。 \\
      \cline{2-3}
      & 資本ストック & 資産関連設備すべてを金額に換算した数値。 \\
      \cline{2-3}
      & 冗長性 & 設備を最低限必要な量より過剰に用意しておくことで、一部の設備が故障し
      てもサービスを継続して提供できるようにシステムを構築すること。 \\
      \hline
      す & スマートフォン & 従来の携帯電話端末の有する通信機能等に加え、高度な情報処理
      機能が備わった携帯電話端末。従来の携帯電話端末とは異なり、利用者が使いたいアプリ
      ケーションを自由にインストールして利用することが一般的。また、スマートフォンはイ
      ンターネットの利用を前提としており、携帯電話の無線ネットワーク( 3G回線等)を通
      じて音声通信網及びパケット通信網に接続して利用するほか、無線 LANに接続して利用す
      ることも可能。 \\
      \cline{2-3}
      & スマートグリッド & 発電設備から末端の機器までを通信網で接続、電力流と情報流を
      統合的に管理することにより自動的な電力需給調整を可能とし、電力の需給バランスを最
      適化する仕組みのこと。 \\
      \cline{2-3}
      & 3G & 「IMT-2000」規格に準拠したデジタル方式の移動通信システム(第 3世代移動通
        信システム)。 NTTDoCoMoの「FOMA」シリーズ、auの「CDMA 1x WIN」シリーズ、
      SoftBankの「SoftBank 3G」シリーズ等が該当。 \\
      \hline
      せ & 0AB-J番号 & 一般的な固定電話に割り当てられる電話番号形式であり、市外局番(
        東京: 03等)を含む 0(ゼロ)から始まる 10桁の電話番号。 \\
      \cline{2-3}
      & センサーネットワーク & 部屋、工場、道路など至る所に埋め込まれたセンサーが周囲
      の環境を検知し、当該情報がユーザや制御機器にフィードバックされるネットワーク。
      \\
      \cline{2-3}
      & センサーデータ & 部屋、工場、道路など至る所に埋め込まれたセンサーによるデータ
      。\\
      \hline
      そ & ソーシャルメディア & 
      ブログ、ソーシャルネットワーキングサービス(SNS)、動画共有サイトなど、利用者が
      情報を発信し、形成していくメディア。利用者同士のつながりを促進する様々なしかけが
      用意されており、互いの関係を視覚的に把握できるのが特徴。 \\
      \cline{2-3}
      & ソリューション & 課題やニーズに対して、情報通信の技術要素(ハードウェア、ソフ
        トウェア、通信回線、サポート要員等)を組み合わせることにより対応すること。(「〜
          サービス」、「〜ビジネス」) \\
      \hline
    \end{tabular}
  \end{center}
\end{table}

\begin{table}[htb]
  \begin{center}
    \caption{用語9}
    \begin{tabular}{|c|c|p{10cm}|}
      \hline
      索引 & 用語 & \multicolumn{1}{c|}{用語解説} \\
      \hline
      た & 第 3世代携帯電話 & 
      「IMT-2000」規格に準拠したデジタル方式の携帯電話。NTT DoCoMoの「FOMA」シリーズ、au
      の「CDMA2000 1x」、「CDMA 1x WIN」シリーズ、SoftBankの「SoftBank 3G」シリーズ等
      が該当。 \\
      \cline{2-3}
      & 第 4世代移動通信システム & 
      第 3世代、 3.9世代移動通信システムの次の世代の移動通信システム( 4G)。高速移動時で
      100Mbps、低速移動時で 1Gbpsの速度を実現するシステム。平成 24年 1月の ITU(国際電気通
        信連合)無線通信総会において、無線規格に関する勧告が承認された。 \\
      \hline
      ち & 地上デジタル放送 & 
      地上の電波塔から送信する地上波テレビ放送をデジタル化したもの。日本では平成 15年
      12月に関東圏・中京圏・近畿圏の三大都市圏で放送が開始された。その後、平成 23年 7
      月 24日に、東日本大震災による影響が大きかった、岩手、宮城及び福島の3県を除く 44
      都道府県で地上アナログ放送が終了し、平成 24年 3月 31日には、東北3県においても地
      上アナログ放送が終了。全国における地上デジタル放送への移行が完了した。 \\
      \hline
      て & 電子商取引 & インターネットを用いて財やサービスの受発注を行う商取引等の総体
      のこと。 \\
      \cline{2-3}
      & テレワーク &
      ICTを活用して、場所と時間にとらわれない柔軟な働き方。企業等に勤務する被雇用者が
      行う雇用型テレワーク(例:住宅勤務、モバイルワーク、サテライトオフィス等での勤務
      )と、個人事業者・小規模事業者等が行う自営型テレワーク(例:SOHO、住宅ワーク)に
      大別される。 \\
      \cline{2-3}
      & データセンタ &
      サーバを設置するために、高度な安全性等を確保して設計された専用の建物・施設のこと。
      サーバを安定して稼働させるため、無停電電源設備、防火・消火設備、地震対策設備等を
      備え、IDカード等による入退室管理などでセキュリティが確保されている。 \\
      \cline{2-3}
      & 電子書籍 & 
      書籍の体裁に近い形で、パソコンや携帯情報端末(PDA)、携帯電話などの ICT機器で読
      めるようにしたデジタルコンテンツ。紙媒体の書籍と異なり、音声や動画を掲載するなど
      、電子書籍特有の表現を行うことが可能。 \\
      \cline{2-3}
      & デジタルサイネージ & 
      日本語では「電子看板」。屋外・店頭・交通機関などの公共空間で、ネットワークに接続
      されたディスプレイなどの電子的な表示機器を使って情報を発信するシステムの総称。設
      置場所や時間帯によって変わるターゲットに向けて適切にコンテンツを配信可能であるた
      め、次世代の広告媒体として注目を集めている。 \\
      \hline
    \end{tabular}
  \end{center}
\end{table}



\begin{table}[htb]
  \begin{center}
    \caption{用語10}
    \begin{tabular}{|c|c|p{12cm}|}
      \hline
      索引 & 用語 & \multicolumn{1}{c|}{用語解説} \\
      \hline
      & テストベッド & 技術や機器の検証・評価のための実証実験、またはそれを行う実験機
      器や条件整備された環境のこと。 \\
      \cline{2-3}
      & 電子掲示板 & インターネット上に開設された掲示板。様々な利用者によって電子化さ
      れた掲示情報の書き込みや閲覧をすることが可能。 \\
      \cline{2-3}
      & 電子署名 & 電子データに付加される電磁的な署名情報であり、付加された電子データ
      の本人性を示すとともに、改ざんが行われていないことを確認できるもの。 \\
      \cline{2-3}
      & 電子カルテ & 診療情報(診療の過程で得られた患者の病状や医療経過等の情報)を電
      子的に保存した診療録もしくはそれを実現するための医療情報システム。\\
      \hline
      と & トラヒック &
      ネットワーク上を移動する音声や文書、画像等のデジタルデータの情報量のこと。通信回
      線の利用状況を調査する目安となる。「トラヒックが増大した」とは、通信回線を利用す
      るデータ量が増えた状態を指す。 \\
      \hline
      は & 8K & 既存のフルハイビジョンの 8倍の画素数を持つ横 
      7,680ドット ×縦 4,320ドット、計 33,177,600
      画素の解像度を持つ高精細液晶パネルや液晶テレビの総称。 \\
      \cline{2-3}
      & テストベッド & 技術や機器の検証・評価のための実証実験、またはそれを行う実験機
      \\
      \hline
      ひ & 
      ビジネスモデル & ビジネスの仕組み。事業として何を行い、どこで収益を上げるのかと
      いう「儲けを生み出す具体的な仕組み」のこと。 \\
      \cline{2-3}
      & ビッグデータ & 
      利用者が急激に拡大しているソーシャルメディア内のテキストデータ、携帯電話・スマー
      トフォンに組み込まれた GPS(全地球測位システム)から発生する位置情報、時々刻々と
      生成されるセンサーデータなど、ボリュームが膨大であると共に、構造が複雑化すること
      で、従来の技術では管理や処理が困難なデータ群。 \\
      \cline{2-3}
      & テストベッド & 技術や機器の検証・評価のための実証実験、またはそれを行う実験機
      \\
      \hline
    \end{tabular}
  \end{center}
\end{table}

\begin{table}[htb]
  \begin{center}
    \caption{用語11}
    \begin{tabular}{|c|c|p{10cm}|}
      \hline
      索引 & 用語 & \multicolumn{1}{c|}{用語解説} \\
      \hline
      ふ & プラットフォーム & 情報通信技術を利用するための基盤となるハードウェア、ソフトウェア、ネットワーク事業等。また、それらの基盤技術。\\
      \cline{2-3}
      & ブログ & Weblog(ウェブログ)の略。ホームページよりも簡単に個人のページを作成し、公開できる。個人的な日記や個人のニュースサイト等が作成・公開されている。 RSS、トラックバック、マッシュアップ、API公開等の技術が情報の流通を円滑にし、モノ等の販売の起点にも広く使われている。\\
      \cline{2-3}
      & 不正アクセス & ID・パスワード等により利用が制限・管理されているコンピュータに対し、ネットワークを経由して、正規の手続を経ずに不正に侵入し、利用可能とする行為。\\
      \cline{2-3}
      & プライバシーポリシー & インターネット上のサービスにおいて、サービス提供者が明らかにするサービスを受ける者の個人情報取り扱い方針のこと。メール -アドレスや通信記録の管理方法などを明らかにする。\\
      \cline{2-3}
      & フィーチャーフォン & スマートフォン以外の従来型携帯電話。\\
      \cline{2-3}
      & フィルタリング & インターネットのウェブページ等を一定の基準で評価判別し、違法・有害なウェブページ等の選択的な排除等を行うソフトウェア。\\
      \cline{2-3}
      & フォトニックネットワーク & 情報を光信号のまま伝達するネットワークのこと。従来の光通信は、ノードは電子回路技術で構成されているが、これを光技術に置き換えて、処理速度の向上や大容量化を達成することが期待されている4G→第4世代移動通信システムの項を参照。\\
      \hline
      へ & ベストプラクティス & 優れていると考えられている事例やプロセス、ノウハウなど。\\
      \hline
      ほ & ポータルサイト & インターネットに接続した際に最初にアクセスするウェブページ。分野別に情報を整理しリンク先が表示されている。\\
      \cline{2-3}
      & ホワイトスペース & 放送用などある目的のために割り当てられているが、地理的条件や技術的条件によって他の目的にも利用可能な周波数。\\
      \cline{2-3}
      & 防災無線 & 地震、火災、天災等の発生時等において、国、地方自治体等の公共機関が円滑な防災情報の伝達等を行うことを目的とした無線通信。\\
      \hline
      ま & マルウェア & malicious softwareの短縮された語。コンピュータウイルスのような有害なソフトウェアの総称。\\
      \hline
    \end{tabular}
  \end{center}
\end{table}


\begin{table}[htb]
  \begin{center}
    \caption{用語11}
    \begin{tabular}{|c|c|p{10cm}|}
      \hline
      索引 & 用語 & \multicolumn{1}{c|}{用語解説} \\
      \hline
      む & 無線 & LANケーブル線の代わりに無線通信を利用してデータの送受信を行う LANシステム。 IEEE802.11諸規格に準拠した機器で構成されるネットワークのことを指す場合が多い。\\
      \hline
      め & 迷惑メール & 受信者の同意を得ずに送信される広告・宣伝目的の電子メール。\\
      \hline
      も & モバイルコンテンツ & モバイルインターネット上で展開されるビジネス(デバイスは、携帯電話端末)。広義では、 iPodやPSPなど携帯型デジタルオーディオ機器や携帯型ゲーム機でのコンテンツのダウンロードなども含む。\\
      \hline
      ゆ & ユニバーサルサービス & 郵便を始め、電話、電気、ガス、水道など生活に欠かせないサービスを、利用しやすい料金などの適切な条件で、誰もが全国どこにおいても公平かつ安定的に利用できるよう提供することをいう。\\
      \cline{2-3}
      & ユニークユーザー & ウェブサイトに訪れた人数。1人で複数回訪れても 1とカウントされる。\\
      \cline{2-3}
      & ユビキタスネットワーク & いつでも、どこでも、何でも、誰でもアクセスが可能なネットワーク環境。なお、ユビキタスとは「いたるところに遍在する」という意味のラテン語に由来した言葉。\\
      \hline
      よ & 4K & 既存のフルハイビジョンの4倍の画素数を持つ横 3,840ドット ×縦 2,160ドット、計 8,294,400画素の解像度を持つ高精細液晶パネルや液晶テレビの総称。\\
      \hline
      り & リテラシー & 本来、「識字力 =文字を読み書きする能力」を意味するが、「情報リテラシー」や「ICTリテラシー」のように、その分野における知識、教養、能力を意味することに使われている場合もある。\\
      \cline{2-3}
      & 臨時災害放送局 & 暴風、豪雨、洪水、地震、大規模な火事その他による災害が発生した場合に、その被害を軽減するために役立つことを目的とし、臨時かつ一時的に開設される放送局。\\
      \hline
      れ & レセプト & 保健医療機関等が療養の給付等に関する費用を請求する際に用いる診療報酬明細書等の通称。急性期病院においては診療内容の詳細情報も含まれる。\\
      \hline
    \end{tabular}
  \end{center}
\end{table}

\end{document}
